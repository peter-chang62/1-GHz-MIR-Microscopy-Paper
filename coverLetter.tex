\documentclass[12pt]{article}
\usepackage{setspace}
\usepackage{fullpage}

\setstretch{1.25}
\begin{document}
\pagenumbering{gobble}
\noindent
Dear editors,
\\\\
In this work, our team demonstrates an important step forward and lays out a roadmap for the application of dual-comb spectroscopy in label free bio-imaging. Hyperspectral imaging with broad spectral coverage ($>$1000 \mbox{$\mathrm{cm^{-1}}$}) in the mid-infrared (\mbox{3 – 12 $\mathrm{\mu m}$}) is a powerful tool for the analysis of biological specimens. The chemical footprints of large biomolecules are encoded in this low energy landscape, that is often accessed through the aid of fluorescent labels hindered by long preparation times and that can alter samples from their native state. However, slow data acquisition and large footprints comprise key drawbacks for mid-infrared imaging platforms that seek to fill this application space and can hinder their widespread deployment when compared to label-based fluorescence microscopy.

In gas-phase sensing, dual-comb spectroscopy (DCS) has been developed into a platform possessing the combination of speed, bandwidth, resolution, stability and small footprint. However, DCS has seen more limited use in condensed-phase imaging, where the broad absorption of large molecules may not leverage the resolution of the frequency comb. In a field with diverse spectroscopic techniques that span from table-top experiments to synchrotron facilities, it remains to be clarified whether DCS can advantageously fit into this important application space. Our work answers this question by situating the performance of the dual-comb spectrometer in the context of existing mid-infrared hyperspectral imaging techniques. We note that the primary challenge in the realization of a DCS scheme is the simultaneous combination of two metrics: high repetition rate $>$ 1 GHz, and broad spectral coverage in the mid-infrared. We make an important step forward in this effort by demonstrating a 1 GHz dual-comb imaging microscope covering $>$1000 \mbox{$\mathrm{cm^{-1}}$} in the \mbox{3 – 5 $\mathrm{\mu m}$} \mbox{(2000 – 3333 \mbox{$\mathrm{cm^{-1}}$})} spectroscopic window. Furthermore, by showing the simple relation between imaging speed and the signal to noise ratio in dual-comb spectroscopy, we elucidate the target metrics for future systems for pushing to the video-rate regime.

We believe our work represents an important step forward in translating the practical advantages of dual-comb spectroscopy to high-speed and broadband hyperspectral imaging in the mid-infrared, particularly for the condensed phase as is relevant to biology. We also believe this article can provide a useful direction for future experiments in this field.
\\\\
Thank you,

Peter Chang, Ragib Ishrak, Nazanin Hoghooghi, Gregory B. Rieker, Rohith Reddy and Scott Diddams on behalf of all coauthors

\end{document}